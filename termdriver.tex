\documentclass{article}

\usepackage{booktabs}
\usepackage{fancyhdr}
\usepackage{float}
\usepackage{fullpage}
\usepackage{graphicx}
\usepackage{helvet}
\usepackage{hyperref}
\usepackage{longtable}
\usepackage{tabularx}
\usepackage{textcomp}
\usepackage{xcolor}

\usepackage{framed}     % These needed for the code formatter
\usepackage{color}
\usepackage{fancyvrb}

% Use helvetica (sans) by default
\renewcommand{\familydefault}{\sfdefault}

% Greenish links
\hypersetup{
  colorlinks=true,
  linkcolor=blue!50!red,
  urlcolor=green!70!black
}

\setlength{\headheight}{40pt}
\setlength{\headsep}{0.2in}

\pagestyle{fancy}
\lhead{\includegraphics[width=0.2\textwidth]{img/logo}}
\chead{TermDriver Datasheet}
\rhead{\thepage}
\cfoot{\textcopyright \the\year \ \ Excamera Labs}
\renewcommand{\headrulewidth}{0.5pt}
\renewcommand{\footrulewidth}{0.5pt}

\newcommand{\heavyline}{\specialrule{1pt}{1pt}{1pt}}
\setlength{\tabcolsep}{20pt}
\newcommand{\png}[2]{
\begin{figure}[H]
\begin{center}
\includegraphics[width=0.75\textwidth]{#1.png}
\caption{#2}
\end{center}
\end{figure}
}

\newcommand{\mach}[1]{\texttt{\textbf{#1}}}
\newcommand{\gap}{\vspace{10pt}}

\input{pyg.tex}

\begin{document}

\begin{figure}[H]
  \centering
  \includegraphics[width=0.75\textwidth]{img/img1}
  \caption{The TermDriver PCB}
\end{figure}

TermDriver listens on a serial line and emulates a text terminal on a
standard VGA connector. It gives embedded microcontrollers a real console.

\begin{itemize}
\item serial input at 115200 bps
\item connects to the Arduino serial output pin, no libraries required
\item supports standard ANSI terminal control codes
\item VGA 16-color output at 1024x768 
\item standard 80x25 and high-resolution 128x48 modes
\item rotated-screen 96x64 mode
\item screen-saver under CPU control
\item input signals 3.3V and 5V compatible
\end{itemize}

\newpage
\tableofcontents
\listoffigures

\newpage
\section{Installation with Arduino}

\begin{enumerate}
\item Disconnect power from the Arduino
\item Attach the TermDriver to the Arduino
\item Connect the VGA plug to the TermDriver and turn on the monitor
\item Apply power to the Arduino. You should see a blank screen with a blinking cursor at top-left
\item Load a sketch on the Arduino that prints text at 115200 baud, like the one below
\end{enumerate}

\newcommand{\eg}[1]{
\begin{framed}
\input{code/#1.inc}
\end{framed}
}

\eg{termdriver-helloworld}

\section{Operation}

\png{img/page1}{sample output at 80x25}

TermDriver monitors the serial line at 115200 baud, and draws any
text on the VGA.
There's nothing to set up or load.
For example this Arduino sketch

\eg{termdriver-counter1}

\noindent
Or this code in plain C

\eg{termdriver-counter2}

\noindent
Gives this output on the VGA:

\png{img/page2}{Counter output}

\subsection{ANSI escape codes}

The following standard
\href{https://en.wikipedia.org/wiki/ANSI_escape_code\#CSI_sequences}{CSI
codes} are supported:

\gap\noindent
\begin{tabularx}{\linewidth}{lX}
\heavyline
Code & Effect \\ \heavyline

ESC {[} \emph{n} A & Cursor up \\

ESC {[} \emph{n} B & Cursor down \\

ESC {[} \emph{n} C & Cursor forward \\

ESC {[} \emph{n} D & Cursor back \\

ESC {[} \emph{r;c} H & Cursor position \\

ESC {[} \emph{n} J & Erase display \\

ESC {[} \emph{n} m & Select graphic rendition \\

ESC {[} s & Save cursor position \\

ESC {[} u & Restore cursor position \\ \heavyline
\end{tabularx}
\gap

In addition the following sequences are specific to TermDriver:

\gap
\noindent
\begin{tabularx}{\linewidth}{lX}
\heavyline
Code & Effect \\ \heavyline

ESC {[} \emph{n} h & 
Set display mode.  0 is 80x25, 1 is 128x48, 2 is 96x64 (rotated)
\\

ESC {[} \emph{n} S & Screen-saver.  0 stops video output, 1 restarts video output
\\ \heavyline
\end{tabularx}
\gap

\noindent
For example this C program displays all available foreground and background colors.

\eg{termdriver-color1}

\subsection{High-resolution modes}

As well as the "classic" 80x25 text mode,
TermDriver also supports a higher density 128x48 mode,
and a portrait orientation 96x64 mode.
Both are very readable because
they match TermDriver's native 1024x768 @ 60 Hz VGA output.

\png{img/moby1}{mode 1 (128x48)}
\png{img/moby2}{mode 2 (96x64)}

\newpage
\section{Hardware}

\png{img/arduino}{Arduino Uno}

TermDriver connects directly to any Arduino or Arduino-compatible.
It requires four connections:

\begin{itemize}
\item \mach{GND}
\item \mach{5V}
\item \mach{RESET}
\item \mach{TX}
\end{itemize}

\noindent
To use another MCU, make the above four connections. Note that RESET is active-low.
The serial protocol on \mach{TX} is 115200 bps, 8 bits, no parity, 1 stop bit.
This is frequently described as \mach{115200-8N1}.
All signaling is 3.3V, but 5V tolerant.

\newpage
\hypertarget{technical-specifications}{}
\hypertarget{technical-specifications}{%
\section{Electrical characteristics}\label{electrical-characteristics}}

\subsection{DC characteristics}
\vspace{10 pt}
{\renewcommand{\arraystretch}{1.2}% for the vertical padding

\begin{tabularx}{\linewidth}{Xcccc}
\heavyline
& min & typ & max & units \\ \heavyline

Supply voltage & 4.0 & 5.0 & 9.0 & V \\ \hline

Supply current & & & & \\
\hspace{10pt} operating & & 25 & & mA \\
\hspace{10pt} screen saver & & 10 & & mA \\ \hline

RESET,RX low voltage & & & 0.6 & V \\ \hline
RESET,RX high voltage & 2.7 &   & 5.8 & V \\ \hline
\end{tabularx}}
\vspace{10 pt}

\subsection{AC characteristics}
\vspace{10 pt}

{\renewcommand{\arraystretch}{1.2}% for the vertical padding
\begin{tabularx}{\linewidth}{Xcccc}
\heavyline
& min & typ & max & units \\ \heavyline

Serial input line speed & & 115200 & & bps \\ \hline

VGA & & & & \\
\hspace{10pt} resolution & & 1024x768 & & pixels \\
\hspace{10pt} vertical sync & & 60.004 & & Hz \\
\hspace{10pt} horizontal sync & & 48.363 & & kHz \\
\hspace{10pt} pixel clock & & 65.000 & & MHz \\ \hline

Cursor blink rate & & 1.875 & & Hz \\ \hline
\end{tabularx}}
\vspace{10 pt}


\end{document}
