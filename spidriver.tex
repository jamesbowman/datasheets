\documentclass{article}

\usepackage{booktabs}
\usepackage{fancyhdr}
\usepackage{float}
\usepackage{graphicx}
\usepackage{helvet}
\usepackage{hyperref}
\usepackage{tabularx}
\usepackage{xcolor}

\usepackage{framed}     % These needed for the code formatter
\usepackage{color}
\usepackage{fancyvrb}

% Use helvetica (sans) by default
\renewcommand{\familydefault}{\sfdefault}

% Greenish links
\hypersetup{
  colorlinks=true,
  linkcolor=blue!50!red,
  urlcolor=green!70!black
}

\setlength{\headheight}{40pt}
\setlength{\headsep}{0.2in}

\pagestyle{fancy}
\lhead{\includegraphics[width=0.2\textwidth]{img/logo}}
\chead{SPIDriver Datasheet}
\rhead{\thepage}
\cfoot{\textcopyright \the\year \ \ Excamera Labs}
\renewcommand{\headrulewidth}{0.5pt}
\renewcommand{\footrulewidth}{0.5pt}

\usepackage{array}
\newcolumntype{L}[1]{>{\raggedright\let\newline\\\arraybackslash\hspace{0pt}}m{#1}}
\newcolumntype{C}[1]{>{\centering\let\newline\\\arraybackslash\hspace{0pt}}m{#1}}
\newcolumntype{R}[1]{>{\raggedleft\let\newline\\\arraybackslash\hspace{0pt}}m{#1}}

\newcommand{\heavyline}{\specialrule{1pt}{1pt}{1pt}}
\newcommand{\png}[2]{
\begin{figure}[H]
\begin{center}
\includegraphics[width=0.75\textwidth]{#1}
\caption{#2}
\end{center}
\end{figure}
}

\newcommand{\mach}[1]{\texttt{\textbf{#1}}}
\newcommand{\gap}{\vspace{10pt}}

\input{pyg.tex}

\begin{document}

\newpage
\begin{center}
\includegraphics[width=0.75\textwidth]{img/spidriver/main}
\end{center}
\tableofcontents
\listoffigures

\section{Overview}

SPIDriver is an easy-to-use tool for controlling SPI devices. It works with Windows, Mac, and Linux, and has a built-in color screen that shows a live logic-analyzer display of all SPI traffic.

\begin{center}
\includegraphics[width=0.75\textwidth]{img/termdriver/block}
\end{center}

\section{Features}
\begin{itemize}
\item live display shows you exactly what it’s doing all the time
\item sustained SPI transfers at 500 Kbps
\item USB line voltage monitor to detect supply problems, to 0.01 V
\item target device high-side current measurement, to 5 mA
\item two auxiliary output signals, A and B
\item dedicated power out lines. two each of GND, 3.3 V and 5 V
\item all signals color coded to match jumper colors
\item all signals are 3.3 V, and are 5 V tolerant
\item uses an FTDI USB serial adapter, and Silicon Labs automotive-grade EFM8 controller
\item also reports uptime, temperature, and running CRC of all traffic
\item all sensors and signals controlled using a simple serial protocol
\item GUI, command-line, C/C++, and Python 2/3 host software provided for Windows, Mac, and Linux
\end{itemize}

\newpage
\section{Installation with Arduino}

\begin{enumerate}
\item Disconnect power from the Arduino
\item Attach the SPIDriver to the Arduino
\item Connect the VGA plug to the SPIDriver and turn on the monitor
\item Apply power to the Arduino. You should see a blank screen with a blinking cursor at top-left
\item Load a sketch on the Arduino that prints text at 115200 baud, like the one below
\end{enumerate}

\newcommand{\eg}[1]{
\begin{framed}
\input{code/#1.inc}
\end{framed}
}

\eg{termdriver-helloworld}

\section{Operation}

\png{img/termdriver/page1}{sample output at 80x25}

SPIDriver monitors the serial line at 115200 baud, and draws any
text on the VGA.
There's nothing to set up or load.
For example this Arduino sketch

\eg{termdriver-counter1}

\noindent
Or this code in plain C

\eg{termdriver-counter2}

\noindent
Gives this output on the VGA:

\png{img/termdriver/page2}{Counter example output}

\subsection{ANSI escape codes}

The following standard
\href{https://en.wikipedia.org/wiki/ANSI_escape_code\#CSI_sequences}{CSI codes}
are supported:

\gap\noindent
\begin{tabularx}{\linewidth}{lX}
\heavyline
Code & Effect \\ \heavyline

ESC {[} \emph{n} A & Cursor up \\

ESC {[} \emph{n} B & Cursor down \\

ESC {[} \emph{n} C & Cursor forward \\

ESC {[} \emph{n} D & Cursor back \\

ESC {[} \emph{r;c} H & Cursor position \\

ESC {[} \emph{n} J & Erase display \\

ESC {[} \emph{n} m & Select graphic rendition \\

ESC {[} s & Save cursor position \\

ESC {[} u & Restore cursor position \\ \heavyline
\end{tabularx}
\gap

In addition the following sequences are specific to SPIDriver:

\gap
\noindent
\begin{tabularx}{\linewidth}{lX}
\heavyline
Code & Effect \\ \heavyline

ESC {[} \emph{n} h & 
Set display mode.  0 is 80x25, 1 is 128x48, 2 is 96x64 (rotated)
\\

ESC {[} \emph{n} S & Screen-saver.  0 stops video output, 1 restarts video output
\\ \heavyline
\end{tabularx}
\gap

\noindent
For example this C program displays all available foreground and background colors.

\eg{termdriver-color1}

\subsection{High-resolution modes}

In addition to standard 80x25 text mode,
SPIDriver supports a higher density 128x48 mode,
and a portrait orientation 96x64 mode.
Both are very readable because
they match SPIDriver's native 1024x768 @ 60 Hz VGA output.

\png{img/termdriver/moby1}{mode 1 (128x48)}
\png{img/termdriver/moby2}{mode 2 (96x64)}

\newpage
\section{Hardware}

\png{img/arduino}{Arduino Uno}

SPIDriver connects directly to any Arduino or Arduino-compatible.
It requires four connections:

\begin{itemize}
\item \mach{GND}
\item \mach{5V}
\item \mach{RESET}
\item \mach{TX}
\end{itemize}

\noindent
To use another MCU, make the above four connections. Note that RESET is active-low.
The serial protocol on \mach{TX} is 115200 bps, 8 bits, no parity, 1 stop bit.
This is frequently described as \mach{115200-8N1}.
All signaling is 3.3V, but 5V tolerant.

\newpage
\section{Raw protocol}

SPIDriver uses a serial protocol to send and receive SPI commands.
Connect to the SPIDriver at 460800 baud, 8 bits, no parity, 1 stop bit (460800 8N1).
Many SPIDriver commands are ASCII, you can control it
interactively from any terminal application that can connect at 460800
baud. For example typing u and s toggles the CS line and ? displays the
status info.
Commands are:

\gap\begin{tabular}{ll}
\hline
\mach{?}        & transmit status information (see below)        \\
\mach{e} $byte$ & echo $byte$       \\
\mach{s}        & select        \\
\mach{u}        & unselect        \\
\mach{a} $byte$ & set A output to 0/1       \\
\mach{b} $byte$ & set B output to 0/1       \\
\mach{x}        & disconnect from SPI bus       \\
\mach{0x80-bf}  & write and read 1-64 bytes       \\
\mach{0xc0-ff}  & write 1-64 bytes        \\ \hline
\end{tabular}\gap

So for example to select, then transfer two bytes 0x12, and unselect, the host sends 5 bytes:

\begin{framed}\begin{Verbatim}
s
0x81
0x12
0x34
u
\end{Verbatim}
\end{framed}

The command \mach{0x81} is a two byte send/receive, so two bytes are returned to the PC.
The status information is always 80 characters, space padded. For example:

{\scriptsize
\begin{framed}\begin{Verbatim}
[spidriver1 DO00QS8D 000007219 4.807 045 25.4 1 1 1 49c1                       ]
\end{Verbatim}
\end{framed}}

The fields are space-delimited:

\gap\begin{tabular}{ll}
\hline
spidriver1      & fixed identifier \\
serial          & serial code identifier \\
uptime          & SPIDriver uptime 0-999999999, in seconds \\
voltage         & USB bus voltage, in volts \\
current         & attached device current, in mA \\
temperature     & junction temperature, in C \\
CS              & CS line state \\
A               & A line state \\
B               & B line state \\
crc             & 16-bit CRC of all input and output bytes (CRC-16-CCITT) \\
\hline
\end{tabular}\gap

\newpage
\hypertarget{technical-specifications}{}
\hypertarget{technical-specifications}{%
\section{Specifications}\label{electrical-characteristics}}

\subsection{DC characteristics}
\vspace{10 pt}
{\renewcommand{\arraystretch}{1.2}% for the vertical padding

\begin{tabularx}{\linewidth}{XC{40pt}C{40pt}C{40pt}C{40pt}}
\heavyline
& min & typ & max & units \\ \heavyline

Voltage accuracy              && 0.01 && V            \\ \hline
Current accuracy              && 5 && mA              \\ \hline
Temperature accuracy          && $\pm$ 2 && C            \\ \hline
MISO & & & & \\
\hspace{10pt}low voltage & & & 0.6 & V \\
\hspace{10pt}high voltage & 2.7 &   & 5.8 & V \\ \hline
Output signal current (SCK, MOSI, CS, A, B)  &&& 8 & mA \\ \hline
Output current        & & & 470 & mA                  \\ \hline
Current consumption   & & 25 & & mA                   \\ \hline

\end{tabularx}}
\vspace{10 pt}

\subsection{AC characteristics}
\vspace{10 pt}

{\renewcommand{\arraystretch}{1.2}% for the vertical padding
\begin{tabularx}{\linewidth}{XC{40pt}C{40pt}C{40pt}C{40pt}}
\heavyline
& min & typ & max & units \\ \heavyline

SPI speed                     &495& 500 &505& Kbps   \\ \hline
Uptime accuracy               && 150 && ppm           \\ \hline
Uptime wrap                   && 31.7 && years        \\ \hline
Startup time & & & 200 & ms \\ \hline
\end{tabularx}}
\vspace{10 pt}

\section{Support information}

Technical and product support is available at
\href{mailto:support@excamera.com}{support@excamera.com}


\end{document}
